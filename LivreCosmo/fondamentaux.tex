%!TEX root = /Users/domaubert/Documents/Lectures/cosmologie/cosmo_main.tex

\chapter{Concepts Fondamentaux}

\section{Définition de l'Univers et de la Cosmologie}
\label{s:fond_def}
 On considère qu'il faut 3 constituants pour définir complètement un Univers:
\begin{enumerate}
\item un contenu en énergie, qui peut être sous diverses formes,
\item un jeu de lois physiques qui régit l'entrejeu des différentes énergies,
\item un espace-temps, qui est la scène sur laquelle cet entrejeu prend place.
\end{enumerate} 
L'enjeu de la \textit{cosmologie} est précisément d'étudier le contenu, les lois physiques et la structure de l'Univers. Par conséquent la cosmologie est liée aux aspects théoriques les plus fondamentaux de la physique mais également à des problèmes astrophysiques et il existe plusieurs manières de faire de la cosmologie. On cherchera par exemple à déterminer quelles sont les lois à l'œuvre dans le cosmos (point 2) , ce qui est davantage du domaine de la physique théorique, ou bien à déterminer les paramètres (point 3)  qui caractérisent la structure spatio-temporelle (e.g. sa forme ou bien son histoire) de l'Univers comme en cosmologie observationnelle. De même comprendre comment les différentes formes d'énergie (matière, lumière, etc...) s'organisent ou évoluent au cours du temps (point 1) (e.g. comment les grandes structures de l'Univers se mettent en place) permet d'avoir un éclairage sur les deux autres aspects du problème cosmologique.

\section{Principe Cosmologique}
L'objectif de la cosmologie est ambitieux et cette ambition n'est pas sans obstacles. L'un des problèmes les plus important est notre incapacité à garantir que ce que nous observons autour de nous reste valable à l'échelle du cosmos, \textit{y compris dans les régions de l'Univers qui nous serons à jamais inaccessibles}. Par exemple, rien ne garantit absolument que la neutralité électrique soit vraie dans tout le Cosmos ou bien que la densité de matière mesurée dans notre Univers observable soit effectivement celle de l'Univers entier. Aujourd'hui, il ne viendrait à l'idée de personne d'attribuer au cosmos une densité égale à celle de la Terre, pour autant c'est peu ou prou ce que nous faisons en cosmologie. 

Pour autant, \textit{nous n'avons pas le choix}. Par définition, il est impossible de connaître les propriétés de régions dont on ne peut extraire de l'information et toujours par définition, il existe de telles régions dans l'Univers. Quelle option reste-t-il à la personne désireuse de faire de la science à l'échelle de l'Univers, si ce n'est de supposer que ce qui nous est accessible est valide dans tout le cosmos ? En l'absence d'une telle hypothèse, il est tout simplement impossible de faire de la science avec l'Univers.

Cette supposition est à la base du \textit{Principe Cosmologique}. Si nous revenons sur les 3 constituants de la section \ref{s:fond_def}, il est aisé de reconnaître que cette supposition implique les contenu universel en énergie doit être le même que celui que nous constatons autour de nous. De même, la structure spatio-temporelle universelle doit être la même que celle que nous constatons dans l'Univers observable. Toutefois cette hypothèse "universaliste" implique également que nous supposons que les même lois de la nature s'appliquent dans tout le cosmos, y compris dans les régions qui nous sont inaccessibles ou pour lesquelles l'information n'est pas extractible. Le \textit{Principe Cosmologique} que nous retiendrons est le suivant : le contenu en énergie de l'Univers, sa structure spatio-temporelle et les lois qui y opèrent sont les même partout et par conséquent sont celles que nous constatons autour de nous. Notons tout de suite que cette universalité des composants s'applique aux échelles pertinentes pour la cosmologie et n'empêche pas des départs locaux aux valeurs universelles.

\subsection{Tautologie}

A ce stade, il nous semble que la cosmologie peut être définie par une tautologie~: \textit{la cosmologie est la science qui met le principe cosmologique à l'épreuve.} Si nous nous retrouvons dans une situation qui reste inexplicable dans ce que nous croyons être le jeu universel de contenu énergétique, spatio-temporel et "législatif" de l'Univers, alors les options sont réduites:
\begin{itemize}
\item soit nous continuons de croire que le principe cosmologique reste valide et c'est le détail de son contenu qui doit être révisé. Par exemple on modifiant les lois de la physique pour qu'elles rendent compte des nouveaux phénomènes tout en continuant à être une bonne description des processus locaux. C'est la voie scientifique standard.
\item soit nous renonçons au principe cosmologique et donc à l'universalité de ses composants et nous renonçons en même temps à l'ambition de décrire tout l'Univers avec la science que nous connaissons.
\end{itemize}

\subsection{Principe cosmologique pragmatique}
Le principe cosmologique est souvent décliné dans une version plus pragmatique qui est la suivante:
\begin{enumerate}
\item la gravitation est correctement décrite par la théorie de la relativité générale d'Einstein,
\item l'Univers est homogène et isotrope.
\end{enumerate}
Le point 2 revient à appliquer au cosmos ce que nous voyons de l'état de l'Univers autour de nous (contenu, géométrie, évolution etc...). Le point 1 revient à définir les lois universelles de la gravité~: une emphase particulière est mise sur cette interaction car elle est la seule qui \textit{in fine} est toujours de portée infinie et donc cosmologique. Par ailleurs, une fois la théorie de relativité générale choisie comme description correcte de la gravité, le point 2 a des conséquences que la relativité générale sait décrire dans le cadre d'étude qu'elle fixe.

\section{Relativité Générale: notions}
La gravitation est centrale à l'étude de la cosmologie car elle est la seule "force" dont l'action ne peut être écrantée et dont la portée soit infinie. De fait, elle est la seule force qui soit effective lorsque des échelles cosmologiques sont abordées. De plus la gravitation peut être décrite comme la manifestation des propriétés de l'espace-temps. Or il s'avère que la structure spatio-temporelle de l'Univers n'est pas triviale comme l'indique par exemple le phénomène d'expansion de l'Univers. Par conséquent l'observation de l'expansion du cosmos nous dit également des choses sur la façon dont la gravitation est à l'œuvre dans le cosmos. Compte tenu du rôle central de la gravitation pour les études cosmologiques, nous allons faire un aperçu de la théorie de la relativité générale et sur son application dans le cadre cosmologique

\subsection{Principe d'équivalence}
Le principe d'équivalence existe sous forme de différentes saveurs. La plus triviale est la suivante, où l'on considère le principe fondamental de la dynamique :
\begin{equation}
\frac{F_z}{m_i}=\ddot z.
\end{equation}
Ici l'on explique que l'accélération suivant la direction z est \textit{proportionnelle} à la force appliquée au système étudié et \textit{inversement proportionnelle} à un coefficient, que l'on nomme \textit{masse inertielle}. Par conséquent, deux systèmes soumis à la même force vont réagir différemment selon la valeur du coefficient $m_i$ qui les caractérisent. Intuitivement on comprend assez rapidement que ce coefficient est lié à la quantité de matière que le système possède, d'où sont qualificatif de masse~: un système fortement chargé en matière réagira moins qu'un système ayant une quantité de matière plus faible. Maintenant considérons l'expression de $F_z$ si cette force se trouve être une force de pesanteur~:
\begin{equation}
F_z=m_g g
\end{equation}
où g est la norme du champ de pesanteur. Cette force $F_z$ est d'autant plus forte que le coefficient $m_g$ est important~: ce coefficient caractérise également la quantité de matière dans le système et un système avec une grande quantité de masse va naturellement avoir une valeur élevée de $m_g$ et donc une pesanteur importante. Cette masse $m_g$ est aussi dénommée \textit{masse grave}.

A ce stade, nous avons donc deux coefficients qui tracent la quantité de matière dans un système physique:le premier $m_i$ lié aux équations de la dynamique, et sans aucune référence à priori à une force de pesanteur (ou de gravitation),  et le second $m_g$ qui lui pour le coup est complètement lié à la présence d'un champ de pesanteur. Il faut alors se rappeler qu'à priori, \textit{rien} ne nous informe d'une quelconque relation quantitative entre les deux et qu'il faut postuler une éventuelle relation entre les masses inertielles et graves.

Toutefois, l'expérience nous indique que tous les systèmes soumis à un champ de pesanteur donné semblent posséder la même accélération $\ddot z$ et ceci quelles que soient leur masse inertielle et grave. Ceci implique que $m_g\sim m_i$. Plutôt qu'une vague égalité, \textit{le principe d'équivalence} stipule une exacte identité entre ces deux quantités:
\begin{equation}
m_g=m_i.
\end{equation}
Cette égalité est fortement suggérée par l'expérience, mais ne peut être démontrée être absolument valide. Selon ce principe, la quantité de matière intervient au travers d'une valeur unique qui est \textit{la masse} $m=m_g=m_i$. On peut noter dès à présent que cette égalité confère à la gravitation un rôle spécial~: la masse \textit{inertielle} est lié à la loi décrivant la dynamique des systèmes, y compris en l'absence de gravitation. Par exemple, une particule chargée dans un champ électrique verra son accélération modulée en $q/m_i$. La force électrostatique, comme toutes les autres forces subit l'impact de ce coefficient d'origine dynamique. D'après le principe d'équivalence, la gravitation quant à elle possède \textit{dans sa propre expression} ce même coefficient, avant expression d'un quelconque problème dynamique. Elle dispose donc d'une façon innée d'une sorte d'élément d'information sur la façon dont la dynamique est régie, information que ne possède pas les autres interactions.

\subsection{Référentiel inertiel}
L'équivalence entre masse grave et inertielle a plusieurs conséquences. Comme déjà mentionné, elle permet de rendre compte de l'universalité de l'accélération de systèmes différents dans un même champ de pesanteur. C'est la fameuse expérience de la tour de Pise où des masses différentes parviennent au sol au même instant car étant accélérée exactement de la même façon. Il en découle également que si l'on fait le choix d'étudier la chute libre de plusieurs objets dans un référentiel lui même en chute libre, ces objets apparaissent tous comme en "flottaison" comme si l'on avait annulé la gravitation dans ce référentiel et ceci même si ils sont tous très différents. A nouveau cela n'est possible que parce que la chute libre dans un champ de pesanteur est la même quel que soit l'objet considéré. Ce type de référentiel est appelé \textit{référentiel inertiel}. Généralement un tel référentiel ne peut être construit que localement et par exemple dans un champ de pesanteur radial, il est possible de construire une collection de référentiel inertiels qui annuleront la gravité localement mais aucun de ces référentiels ne peut annuler globalement le champ de pesanteur. A ce titre il est usuel de se référer à ce type de référentiel sous l'appellation \textit{référentiel localement inertiel}. Cette possibilité d'annuler localement la gravité est intrinsèquement lié à l'égalité $m_i=m_g$. Ce lien est si fort que l'on va considérer que \textit{la possibilité d'annuler la gravitation par un choix approprié de référentiel est aussi un énoncé du principe d'équivalence}. 

Notez qu'à l'aide d'un choix approprié de référentiel on peut également créer de la gravité. Si des systèmes libres sont placés dans un référentiel uniformément accéléré, ils percevront leur mouvement relatif comme induit par une force de gravitation. \textit{De façon générale il est impossible localement de distinguer un référentiel uniformément accéléré d'un champ de gravitation.}

\subsection{Qu'est-ce que la gravitation ?}
A nouveau, on peut créer ou "détruire" un champ gravitationnel par un choix approprié de référentiel. Or un référentiel pourrait se résumer à un ensemble de règles et d'horloges ayant un certain comportement donc à un objet qui n'a qu'une nature géométrique et non pas physique. A partir de cette constatation, il est tentant de considérer alors que la gravitation n'est qu'une manifestation géométrique de l'espace dans lequel évoluent les systèmes. C'est le choix que fait la théorie de la relativité générale (RG par la suite), où la gravitation n'est pas une force à proprement parler mais davantage une manifestation de la géométrie de l'espace-temps. En RG, les systèmes se déplacent librement dans une géométrie qui, si elle est non triviale, produit des effets qui peuvent être interprétés comme le produit d'une interaction. A ce titre on peut dire que \textit{la gravitation n'existe pas}, elle n'est qu'une interprétation d'un effet de nature fondamentalement géométrique. 

Ainsi, on verra par la suite que c'est la courbure de cette géométrie qui produit les effets gravitationnels. A l'inverse, une géométrie sans courbure, i.e. une géométrie plane, produit un environnement sans effets de gravitation. Or il est aisé d'imaginer qu'il est toujours possible de trouver \textit{localement} un jeu de coordonnées qui rendent une géométrie plane, i.e. une transformation \textit{locale} qui permette de détruire la gravitation. Par analogie, une fonction régulière peut être approximée comme une collection de tangentes sur lesquelles il n'y a pas de courbures et qui localement sont des représentations exactes de la fonction originale. C'est une manifestation du principe d'équivalence~:\textit{il est toujours possible de trouver une transformation locale qui rende la géométrie de l'espace-temps plane}. Plus précisément c'est parce que l'on considère la gravitation comme étant de la géométrie que le principe d'équivalence se trouve naturellement réalisé.

Cette vue de la gravité comme géométrie de l'espace-temps est l'école classique d'interprétation de la RG. Elle n'est toutefois pas sans poser problème par exemple si l'on cherche à quantifier la gravitation. Si cette dernière est pure géométrie alors on peut arguer qu'il n'y a rien à quantifier et par exemple il n'y a pas de raison à priori d'invoquer l'existence d'un boson porteur de l'interaction, coupant court à toute tentative de quantification et donc d'unification. Si l'on estime que la quantification est nécessaire alors la vision géométrique n'est qu'une interprétation, certes très puissante, d'un processus qui n'est pas pure géométrie. Dans ce cadre le principe d'équivalence devient prééminent: il faut le supposer réalisé par un mécanisme encore inconnu et sa réalisation conduit à une possible interprétation géométrique. L'approche classique vue précédemment raisonne de façon inverse.

\subsection{Espace-temps et Métrique}
La théorie de la relativité générale décrit la gravitation comme une manifestation de la géométrie de l'espace-temps. Cet espace temps possède 4 dimensions, une de temps et trois d'espace. L'outil mathématique permettant de décrire cette géométrie est la géométrie différentielle. Une quantité centrale est \textit{la métrique}~: fondamentalement elle est l'outil qui permet de calculer des distances (des produits scalaires) dans une géométrie arbitraire. En notation d'Einstein, le calcul d'une distance s'écrit de la façon suivante :
\begin{equation}
ds^2=g_{\mu\nu}dx^\mu dx^\nu,
\label{e:scal}
\end{equation}
où les indices $\mu,\nu$ courent sur les indices des coordonnées, $ds^2$ est un scalaire donnant la distance couverte par un intervalle $(dx^0,dx^1,dx^2,dx^3)$. La quantité $g_{\mu,\nu}$ est la métrique et permet de relier la distance aux composantes de l'intervalle. Si la géométrie est plane, la métrique aura une certaine forme et si la géométrie est courbe et complexe, cette expression sera différente. En géométrie euclidienne plane à 3D le calcul de distance est le suivant:
\begin{equation}
ds^2=(dx^1)^2+(dx^2)^2+(dx^3)^2.
\end{equation} 
En géométrie de Minkowski, correspondant à l'espace-temps plat utilisé par la relativité restreinte, le calcul devient
\begin{equation}
ds^2=(dx^0)^2 -((dx^1)^2+(dx^2)^2+(dx^3)^2)
\end{equation} 
où $dx^0=cdt$ est la composante liée au temps. L'expression générale est elle donnée par Eq. \ref{e:scal}.

Cette métrique synthétise la structure spatio-temporelle de la \textit{variété} que l'on cherche à étudier. Pour faire de la cosmologie, il faut ainsi se doter d'une telle métrique, la plus à même de représenter ce que l'on croit être les caractéristique générique du cosmos.

\subsection{Métrique de Friedman-Robertson-Walker}
En se rappelant l'énoncé du principe cosmologique pragmatique, la métrique devant servir à décrire l'Univers doit refléter les propriétés d'homogénéité et d'isotropie. La métrique la plus générique satisfaisant ces contraintes est la métrique de Friedman-Robertson-Walker (FRW). A l'aide de celle-ci, l'intervalle de distance 4D est donné par:
\begin{equation}
ds^2=c^2dt^2-a(t)^2(\frac{dr_0^2}{1-Kr_0^2}+r_0^2d\theta^2+r_0^2\sin^2\theta d\phi^2).
\label{e:FRW}
\end{equation}
On note que cette métrique fait usage d'un système de coordonnées sphériques $(r_0,\theta,\phi)$ pour sa partie espace. On note également que par rapport aux exemples Euclidien et Minkowskien, FRW couple explicitement les parties temporelles et spatiales, via le facteur d'expansion $a(t)$.  

$r_0$ est une \textit{distance comobile}~: c'est une coordonnée de nature spatiale et indépendante du temps. Ici elle désigne une distance radiale prise à partir de l'origine du système de coordonnées et est également prise en compte pour calculer les contributions à l'intervalle des séparations angulaires $d\theta$ et $d\phi$. Le paramètre $K$ est un paramètre \textit{de courbure} auquel on peut éventuellement associer un rayon de courbure $R_K=K^{-1/2}$.

Si l'on considère deux évènements sur une même ligne de visée (donc avec $d\theta=d\phi=0$), l'intervalle peut se synthétiser sous une forme
\begin{equation}
ds^2=c^2dt^2-dr^2,
\end{equation}
avec 
\begin{equation}
dr=\frac{a(t)dr_0}{\sqrt{1-Kr_0^2}}.
\label{e:phydist}
\end{equation}
Ici $dr$ désigne la distance \textit{physique} de la partie spatiale (donc 3D) de l'intervalle que nous étudions. Cette distance dépend du temps, donc de l'instant considéré, et est modulé par une éventuelle courbure. A proximité de l'origine ($r_0\rightarrow 0$) ou dans des régimes de très faible courbure ($R_K\rightarrow\infty$ ou $K\rightarrow 0$), distance physique et distance comobile sont directement reliées par :
\begin{equation}
dr=a(t)dr_0.
\end{equation}
Là où la distance comobile était une quantité statique, la distance physique est une quantité évolutive, dont la dépendance temporelle est encodée par le facteur d'expansion $a(t)$. Notons que si l'on considère le parcours d'un photon sur un intervalle infinitésimal ($t$ est alors quasi constant et $r_0<<1$) alors $dr=cdt$ et la distance physique est celle effectivement parcourue par le rayon lumineux. Si la géométrie est plane, $K=0$, et la distance de parcours lumineux \textit{dans ce cas précis} vaut $D_L=dr=a(t)dr_0$.  Si toutefois la courbure de l'espace est positive et non nulle, alors $ D_L>a(t)dr_0$ et dans le cas d'une courbure négative $ D_L<a(t)dr_0$~: de façon générale, la courbure induit un départ de la distance physique par rapport à une fonction simple de la distance comobile.

Considérons à nouveau la distance physique en se plaçant à un instant donné et en calculant $r(t)$ entre 2 points A et B, séparés d'une distance comobile $R_0$ le long d'une direction avec $(\theta,\phi)$ donnés. En pratique il s'agit d'intégrer l'équation \ref{e:phydist} tout en considérant $t$ constant. On a alors
\begin{eqnarray}
&&a(t)R_0\\
r(t)&=&a(t)R_K \arcsin(R_0/R_K)\\
&&a(t)R_K \mathrm{argsh}(R_0/R_K)
\end{eqnarray}
pour une courbure $R_K=K/|K|^{3/2}$ respectivement nulle, positive et négative. Notons que les termes de courbure sont indépendants du temps. Les trois solutions peuvent être synthétisées sous la forme d'une équation unique $r(t)=a(t)S_K(r_0)$. A nouveau, les distances physiques sont des fonctions du temps qui ne dépendent que de la variation temporelle du facteur d'expansion $a(t)$. Si $a$ est un fonction croissante, \textit{toutes} les distances augmentent avec le temps et on parle d'Univers en expansion. A l'inverse, $a$ peut être une fonction décroissante du temps, auquel cas l'Univers est en contraction.

Pour finir, mentionnons que les distances physiques aujourd'hui sont obtenues en prenant la valeur du facteur d'expansion prise aujourd'hui. Ces types de quantités mesurées aujourd'hui sont notées par convention avec l'indice 0. Par exemple le temps qui a pu s'écouler depuis le Big-Bang jusqu'à aujourd'hui est noté $t_0$. De même le facteur d'expansion aujourd 'hui est noté $a_0=a(t_0)$.
\textit{Par convention} le facteur d'expansion est normalisé à cette valeur actuelle et 
\begin{equation}
a_0=a(t_0)=1.
\end{equation}
Il en découle que 
\begin{equation}
r(t_0)=S_K(r_0).
\end{equation}
et dans le cas sans courbure, on constate que la distance physique mesurée aujourd'hui est égale à la distance comobile $r(t_0)=r_0$.

\subsection{Facteur d'expansion, Loi de Hubble}
La distance physique $r(t)$ entre deux points peut être simplement dérivée au cours du temps.
\begin{equation}
\dot r(t)= \dot a S_K(r_0) =\frac{\dot a}{a}r(t).
\end{equation}
On définit le \textit{paramètre de Hubble} comme la fonction dépendante du temps donnée par :
\begin{equation}
H(t)=\frac{\dot a}{a}.
\label{e:hubble}
\end{equation}
A l'aide de cette nouvelle fonction, le taux de variation des distance peut s'exprimer comme une fonction linéaire de la distance, \textit{la loi de Hubble}:
\begin{equation}
\dot r(t) = H(t) r(t).
\label{e:hubble2}
\end{equation}
Si $a$ est une fonction croissante, les distances physiques varient d'autant plus qu'elles sont importantes.

Plusieurs remarques peuvent être faites à propos des Eqs \ref{e:hubble} et \ref{e:hubble2}.  La première est que le paramètre de Hubble n'est pas une constante temporelle, en revanche sa valeur ne dépend pas du point de l'espace considéré : on peut considérer que c'est une constante \textit{spatiale} de valeur donnée dans tout l'Univers à un instant donné. Suivant une convention générique en cosmologie sa valeur actuelle est notée avec un indice 0 et vaut
\begin{equation}
H_0 = 67 \mathrm{km/s/Mpc}.
\end{equation}
On peut constater au vu des unités employées et au vu de la structure de l'équation \ref{e:hubble}, que le paramètre de Hubble a la dimension de l'inverse d'un temps. On définit ainsi le temps de Hubble par $t_H=H^{-1}$~: on verra par la suite que ce temps de Hubble est une bonne approximation de l'âge de l'Univers. La seconde remarque porte sur le caractère linéaire de l'équation \ref{e:hubble2}~: on peut montrer que cela permet de conserver le caractère isotrope et homogène des points de vue, comme exigé par le principe cosmologique. Si la loi avait été constante ($\dot r \sim r^0$) ou bien quadratique ($\dot r\sim r^2$), l'homogénéité aurait été perdue. La dernière remarque porte sur le fait que la loi donnée par l'Eq. \ref{e:hubble2} donne l'impression que des récessions supraluminique sont autorisées ($\dot r>c$)~: il s'avère que cela est exact, mais le taux de variation de distance calculé ici n'implique pas de déplacement par rapport à un référentiel local inertiel (qui lui serait limité par $c$) mais se rapporte à une dilatation  même de l'espace~: dit rapidement, ceci n'est pas la vitesse d'un corps en déplacement. Par exemple, insistons sur le fait que cette dilatation ne permet pas de transmettre d'information par exemple.

\subsection{Décalage vers le rouge}
Nous venons de voir que la forme de la métrique FRW conduit naturellement à la Loi de Hubble. Dans le même ordre d'idée, la métrique FRW conduit naturellement à une modification de la perception des intervalles temporels. Si l'on considère par exemple l'émission d'un photon depuis un point E jusqu'à sa réception au point R, l'intervalle séparant les deux évènements est nul, comme c'est toujours le cas pour une particule sans masse. Soit $E$ l'origine du système de référence, alors E et R se trouvent sur un seul et même rayon-vecteur ($d\theta=d\phi=0$) et FRW permet d'écrire:
\begin{equation}
\int_{t_E}^{t_R}c\frac{dt}{a(t)}=\int_{r_E}^{r_R}\frac{ dr_0}{\sqrt{1-Kr_0^2}}.
\end{equation}
Considérons un 2ème photon qui va effectuer le même parcours mais en étant émis à l'instant $t_E+\delta_E$ et reçu à l'instant $t_R+\delta_R$. Dans un espace-temps statique, on s'attend à obtenir $\delta_E=\delta_R$. Pour ce 2ème photon FRW permet d'écrire:
\begin{equation}
\int_{t_E+\delta_E}^{t_R+\delta_R}c\frac{dt}{a(t)}=\int_{r_E}^{r_R}\frac{ dr_0}{\sqrt{1-Kr_0^2}}.
\end{equation}
Notons que les bornes d'intégration du second membre restent inchangées, le 2nd photon passant par les 2 même endroits en coordonnées comobiles: par conséquent les 2 intégrales temporelles des 2 photons sont identiques permettent d'écrire la relation :
\begin{equation}
\int_{t_E}^{t_E+\delta_E}c\frac{dt}{a(t)}=\int_{t_R}^{t_R+\delta_R}c\frac{dt}{a(t)}.
\end{equation}
Si l'on suppose que les délais temporels $\delta_E$ et $\delta_R$ sont suffisamment petits par rapport au temps typique d'évolution du facteur d'expansion $a$ alors l'on obtient que le délai mesuré à la réception diffère du délai à l'émission:
\begin{equation}
\delta_R=\frac{a(t_R)}{a(t_E)} \delta_E.
\end{equation}
Cette relation est valable pour tous les délais~: si la métrique n'est pas statique $a(t_E)\neq a(t_R)$, les délais sont modifiés. Par exemple on constate que les courbes de lumières de supernovæ sont affectées par cet cette modification des délais. De même les flux de photons subissent une "dilution" cosmologique à cause d'une modification de la durée devant s'écouler entre 2 photons. Enfin la \textit{longueur d'onde} $\lambda=cT$ est directement proportionnelle à une durée (la période). La longueur d'onde d'un rayonnement électromagnétique reçu aujourd'hui est donc aussi affectée~:
\begin{equation}
\lambda_0=\frac{\lambda_E}{a(t_E)}
\end{equation}
où on a utilisé la convention $a(t_0)=1$. Il en découle que le décalage vers le rouge est donné par
\begin{equation}
z = \frac{\lambda_0-\lambda_E}{\lambda_E}=\frac{1}{a_E}-1.
\end{equation}
Si le facteur d'expansion était plus petit dans le passé (comme attendu pour un Univers en expansion), le décalage vers le rouge (ou \textit{redshift}) est positif ou nul. Notons qu'à aucun moment il n'est fait mention de vitesse de déplacement de source ou de récepteur. Le seul effet contributeur est celui d'un espace temps non statique qui donne un effet similaire à un effet Doppler mais qui en aucun cas ne nécessite que la source ou l'émetteur possède une vitesse non nulle.



\subsection{Source de la gravitation}
A ce stade, l'élément essentiel qui reste à être précisé est la loi qui régit l'évolution du facteur d'expansion et par extension celle de la métrique FRW. Plus généralement quelle sont les lois qui permettent de relier la métrique $g_{\mu\nu}$ (donc la structure de l'espace-temps) aux sources de la gravitation ? Ces relations sont connues et portent le nom d'équation d'Einstein. Leur détermination exacte est une démarche relativement complexe mais le cheminement logique qui aboutit à leur obtention peut être aisément décrit. 

Un bon point de départ est l'équation de champ de la gravitation Newtonienne, à savoir l'équation de Poisson. Elle relie la densité de matière $\rho$ au potentiel gravitationnel $\phi(x)$, qui est une fonction simple du champ de gravitation~:
\begin{equation}
\Delta \Phi =\rho.
\label{e:poisson}
\end{equation}
Clairement, le potentiel gravitationnel a pour source la densité de matière. En RG, l'objectif est d'obtenir une équation de champ analogue mais reliant $g_{\mu\nu}$ à ses sources. Disons d'emblée que dans le régime des champs faibles( dans lequel la gravitation Newtonienne s'applique) métrique et potentiel gravitationnel sont directement relié.

Quelles sont à priori les sources de la métrique en RG ? Une première tentative peut être effectué en considérant directement la densité de matière $\rho$ comme source~: c'est le cas quand les champs sont faibles et la matière est source de gravitation donc de géométrie. Toutefois on sait également que la masse en tant que tel ne joue pas de rôle central dans les théories relativistes, c'est davantage \textit{l'énergie} $E$ qui joue un rôle de premier plan. 

La \textit{densité} énergie serait-elle cette source ? Il est clair qu'elle doive jouer un rôle, par exemple au travers de la densité d'énergie de masse $\rho c^2$, ce qui reviendrait simplement à considérer la contrepartie énergie de la source de gravité Newtonienne. La difficulté de cette option réside dans le fait que l'énergie n'est pas directement une quantité fondamentale, y compris en relativité restreinte~: l'énergie n'est qu'une composante d'un concept plus vaste qui est l'énergie-impulsion. Par conséquent si l'énergie source la gravité en RG, c'est au travers de l'énergie-impulsion et non seule. Plus précisément c'est au travers d'une \textit{densité} d'énergie-impulsion.

Si on note l'énergie impulsion $(P^0,P^1,P^2,P^3)$ où $P^0=E$ (en supposant $c=1$), on arrive à des expressions de densités de type:
\begin{equation}
\frac{dP^\mu}{dx^\alpha dx^\beta dx^\gamma}.
\end{equation}
Notons que $\alpha,\beta,\gamma$ désigne toutes les coordonnées disponible, \textit{temps} compris. Bien sûr la densité d'énergie sus-mentionnée fait partie des sources:
\begin{equation}
\frac{dE}{dx dy dz},
\end{equation}
tandis que l'expression suivante est tout aussi légitime pour constituer un terme source de la gravité:
\begin{equation}
\frac{dP_x}{dtdydz}.
\end{equation}
Cette dernière expression désigne une force ($dP_x/dt$) par une unité de surface à $x$ constant, à savoir une pression suivant la direction $x$. Notons qu'une pression est homogène à une densité d'énergie.

En généralisant cette expression, on obtient le tenseur des contraintes ou \textit{tenseur énergie-impulsion}
\begin{equation}
T^{\mu\delta} = \frac{dP_\mu}{dx^\alpha dx^\beta dx^\gamma}.
\end{equation}
obtenu en considérant la densité de la composante $\mu$ de $P$ sur la "surface" de coordonnées $x^\delta$ constante. C'est la source de la métrique, qui généralise la densité de masse du cas newtonien.  Comme mentionné ci dessus, la densité d'énergie a donc un rôle à jouer mais la également la pression, de même que le cisaillement. Par la suite la contribution d'un type d'énergie à la dynamique de l'espace-temps devra s'évaluer au vu de toutes ces quantités. 

Notons que $T$ obéit à une équation analogue à généralisation de la conservation de l'énergie dans un espace-temps quelconque : 
 \begin{equation}
 \nabla_\nu T^{\mu\nu}=0,
 \label{e:divT}
 \end{equation}
 cette relation est tensorielle et non triviale et définit une "conservation" globale de l'objet $T$. La densité d'énergie est l'une des composantes de cet objet et n'est pas généralement conservé \textit{individuellement}, ne serait-ce que parce que l'expression de la densité d'énergie dépend du système de coordonnées choisi.

A ce stade, le terme source de la dynamique de l'espace-temps est connu, reste à expliciter l'opérateur différentiel qui agit sur la métrique et qui joue un rôle analogue au laplacien de l'équation \ref{e:poisson}. En d'autres termes quel est le tenseur $G$ , fonction de la métrique qui permette d'avoir:
\begin{equation}
G^{\mu\nu}=T^{\mu\nu}.
\end{equation}
Sans rentrer dans les détails, on cherchera à obtenir un opérateur différentiel du second ordre, tout comme pour l'équation de Poisson newtonienne. De plus il faut que l'équation \ref{e:divT} soit satisfaite par $G$. L'opérateur différentiel du second ordre le plus simple permettant de satisfaire à cette contrainte est le \textit{tenseur d'Einstein}:
\begin{equation}
G^{\mu\nu}=R^{\mu\nu}-\frac{R}{2}g^{\mu\nu}
\label{e:einstein}
\end{equation}
où $R^{\mu\nu}$ est le tenseur de Ricci ou \textit{tenseur de courbure} et qui dépend de dérivées de second ordre de $g^{\mu\nu}$. $R$ est la trace de ce tenseur et est appelé aussi scalaire de courbure ou scalaire de Ricci.

Pour résumer, l'équation de champ de la gravité dans le cadre de la relativité générale décrit le comportement de la métrique en fonction du contenu local en énergie, via :
\begin{equation}
G^{\mu\nu}=T^{\mu\nu}.
\label{e:RG}
\end{equation}
Plusieurs remarques peuvent être faites à ce stade. La première est que nous sommes passé d'une équation scalaire dans le cas newtonien à une équation tensorielle avec dans l'absolu $ 4\times 4=16$ équations couplées à résoudre. Toutefois le tenseur d'énergie impulsion est symétrique et au final seules 10 équations indépendantes restent réellement dans le cas générique. De plus si les sources d'énergies possède un certain degré d'isotropie (comportement de fluide parfait par exemple) les composantes anisotropes de la pression ne rentrent pas en jeu et le système peut encore être réduit. La seconde remarque porte sur la structure du tenseur d'Einstein (Eq. \ref{e:einstein}). On constate ainsi que l'opérateur est non linéaire  en métrique (via par exemple $\frac{R}{2}g^{\mu\nu}$), ce qui rend l'équation de champ particulièrement complexe à résoudre dans le cas général. On constate également que la métrique $g^{\mu\nu}$ est elle-même source de courbure. Ceci peut être généralisé de la façon suivante en écrivant l'équation \ref{e:RG} avec un terme supplémentaire:
\begin{equation}
G^{\mu\nu}=T^{\mu\nu}+\Lambda g^{\mu\nu}
\label{e:modifeins}
\end{equation}
où $\Lambda$ est un scalaire constant dans l'espace-temps. Compte tenu de $\nabla_\mu g^{\mu \nu}=0$, il en découle que \ref{e:modifeins} reste valable et dans ce cas également la métrique est source de courbure. Notons que le scalaire $\Lambda$ introduit dans l'équation Eq. \ref{e:modifeins} s'interprète comme une \textit{constante cosmologique}.

\section{Equations de Friedmann}
Comme indiqué dans la section précédente, l'hypothèse d'un Univers homogène et isotrope conduit à une forme de la métrique donnée par FRW. Considérant de plus que la relativité générale fournit une bonne description de la dynamique de l'espace temps, on est en principe capable de résoudre les équations d'Einstein pour une métrique FRW. Le résultat de cette opération est \textit{l'équation de Friedmann}. Elle repose sur la RG, sur la métrique de FRW et considère que le contenu en énergie de l'Univers est bien décrit par des "fluide" parfaits, de pression isotrope et sans viscosité (donc sans contraintes).

L'équation de Friedmann est une relation scalaire qui relie le facteur d'expansion $a(t)$, qui restait la dernière quantité non précisée de la métrique, au contenu en énergie de l'Univers~:
\begin{equation}
\frac{\ddot a}{a}=-\frac{4\pi G}{3c^2}(\rho c^2 +3 P),
\label{e:friedmann}
\end{equation}
où $\rho c^2$ est la densité d'énergie de l'Univers et $P$ sa pression, qui sont toutes deux fonctions du temps et donc de $a$. On constate que comme prévu, la dynamique de l'espace temps est lié à son contenu énergétique. Notons que \textit{à priori} le second terme est défini positif, donc la dynamique de l'Univers semble être décélérée $\ddot a<0$. Un modèle d'Univers se définira à partir d'une solution de cette équation différentielle, qui elle même dépendra du contenu en énergie. C'est l'objet du chapitre suivant.