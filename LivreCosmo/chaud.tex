%!TEX root = /Users/domaubert/Documents/Lectures/cosmolog/cosmo_main.tex

\chapter{l'Univers Chaud}
Dans ce chapitre nous allons étudier de premières bases se rapportant à l'Univers quasi-primordial âgé de quelques minutes au plus. Durant cette période, la dynamique de l'Univers est régie par les espèces relativistes, avec une dépendance temporelle du facteur d'expansion en $a(t)\sim\sqrt{t}$~: durant ces époques les influences respectives de la matière et de la constante cosmologique sont faibles. Ces premiers instants sont proches du Big-Bang et l'Univers s'y trouve dense et chaud~: ces conditions sont propices aux interactions entre atomes et particules subatomiques. C'est durant cette époque que les abondances des particules reliques et des éléments légers sont fixées~: c'est de ces abondances dont nous allons discuter dans ce chapitre.

\section{Equilibre \& Gel de Réactions}
Les deux concepts fondamentaux des processus qui règlent les abondance sont les notions \textit{d'abondance à l'équilibre} et de \textit{gel des réactions}. Dans notre cas, le terme d'abondance désigne la densité numérique d'une espèce atomique, subatomique, isotopique, etc... Par exemple l'abondance des atomes d'hydrogène se note $n_H$ et s'expriment en atomes par $m^3$. Les réactions qui permettent de modifier ces abondances font généralement intervenir d'autres réactifs~: la photoionisation par exemple se caractérise par la réaction suivante:
\begin{equation}
n_H+\gamma \leftrightarrow n_{H+} + e^-,
\end{equation}
et dépend non seulement de l'abondance des atomes d'hydrogène mais également de celle du nombre de photons ionisants. Toutefois dans un très grand nombre de cas, la variation d'une espèce peut s'écrire comme ne dépendant que de sa propre abondance \textit{à l'équilibre} $n_e$ et d'un taux de réaction $\Gamma$ constant. Soit $n$ une abondance quelconque, son évolution pourra être suivie par une équation différentielle du type:
\begin{equation}
\frac{dn}{dt}=-\Gamma (n-n_e).
\end{equation}
Celle-ci est simple à comprendre. Si une abondance est déjà à l'équilibre $n=n_e$ et son abondance, par définition, ne varie pas. Si son abondance est supérieure à l'équilibre, le taux de réaction va agir comme une force de rappel, traduisant de fait une tendance à favoriser les réactions de destruction de l'espèce étudiée. A l'inverse si l'abondance est en déficit par rapport à l'équilibre, les réactions vont avoir tendance à la rétablir à des valeurs plus élevées. Notons que  l'inverse du taux de réactions fournit un temps caractéristique de retour à l'équilibre $t_e=\Gamma^{-1}$.

Toutefois il existe une autre manière de faire varier la densité numérique d'une espèce dans le contexte qui est le nôtre: il s'agit de la dilution cosmologique, déjà rencontrée dans le chapitre précédent. En effet, compte tenu de l'expansion de l'Univers, si l'on dispose d'un certain nombre de particules d'un type donné dans un certain volume, sa \textit{densité} va évoluer même en l'absence de réaction (c'est à dire de destructions/créations). Ainsi la densité numérique d'une espèce donnée varie cosmologique de la façon suivante:
\begin{equation}
n=\frac{n_0}{a^3},
\end{equation}
ce qui ne fait que traduire l'équation différentielle suivante:
\begin{equation}
\frac{dn}{dt}=-3Hn,
\end{equation}
où $H$ est le paramètre de Hubble usuel, fonction du temps ou du paramètre d'expansion $H=\dot a/a$. Comme déjà indiqué, le temps de Hubble $t_H=H^{-1}$ fournit le temps caractéristique d'évolution significative des distances dans le cosmos.

Dans le cas cosmologique général, les deux procédés se superposent et l'abondance d'une espèce arbitraire est régie par une équation de type:
\begin{equation}
\frac{dn}{dt}=-3Hn-\Gamma (n-n_e),
\label{e:reac}
\end{equation}
cette équation demande en toute généralité d'être résolue numériquement. Toutefois 2 cas limites se détachent facilement:
\begin{itemize}
\item si $H\gg \Gamma$ : l'expansion est beaucoup plus efficace que les réactions. C'est un régime où le 2nd terme de l'équation \ref{e:reac} peut être négligé, on retrouve $n\sim a^{-3}$ et le nombre de particules dans un volume en expansion donné est \textit{constant}. On dit que l'espèce est \textit{gelée}.
\item si $H\ll \Gamma$ : on peut négliger la dilution cosmologique et les temps de retour à l'équilibre sont très courts. L'abondance est celle de l'équilibre, qui est éventuellement une fonction du temps $n\sim n_e(a)$.
\end{itemize}

En règle générale $H$ et $\Gamma$ sont tous deux fonctions du temps et $\Gamma$ a tendance à dominer au début de l'histoire de l'Univers (qd les densités et températures sont très élevées) pour être ensuite dominé par $H$. Par conséquent l'histoire typique de l'abondance d'une espèce suit d'abord celle de l'équilibre avant d'être gelée et n'être plus modifiée que par la dilution cosmologique. Cette transition porte le nom de "gel" ou \textit{freeze-out} en anglais.

\section{Statistique d'un gaz}
La question qui se pose à présent est celle de déterminer l'abondance à l'équilibre d'une espèce (hydrogène, photons, neutrinos, etc....). Celle ci nous est donnée par la physique statistique.

On se place dans le cas simple d'une particule libre, auquel cas son énergie ne dépend que de son impulsion $\vec p$, ou bien de façon équivalente que de son vecteur d'onde $\vec k =\vec p /\hbar$:
\begin{equation}
E^2=p^2c^2+m^2c^4=\hbar^2 c^2 k^2 +m^2c^4
\end{equation}
 Plus précisément l'énergie d'une particule ne dépend que de la norme $k$ du vecteur d'onde. Par conséquent, l'ensemble des points dans l'espace des $\vec k$ qui fournissent une énergie donnée sont à l'intérieur d'une coquille de rayon $k$. De plus, le peuplement de cet espace est quantifié : en effet les nombres d'ondes accessibles (i.e. les impulsion accessibles) doivent être de la forme $\vec k = (n_x,n_y,n_z) 2\pi/L $ où $L$ désigne la taille de la "cuve" dans laquelle s'effectue l'étude et où le triplet est un triplet de valeurs entières. Par conséquent, une particule ne peut se trouver que sur les nœuds d'une maille pavant cet espace. 

Ces considérations nous permet d'évaluer le \textit{nombre d'états accessibles à une énergie E donnée}. Ce nombre est donné par le rapport entre le volume de l'espace des $\vec k$ à énergie $E$ donnée (la coquille) et le volume occupé par un état unique (le volume de la maille). On obtient alors le nombre d'états accessible à une particule libre d'énergie $E$ à $dE$ près:
\begin{equation}
N(E)dE=\frac{4\pi k^2 dk}{(2\pi/L)^3}.
\label{e:densetat}
\end{equation}

L'équation \ref{e:densetat} n'est pas suffisante pour calculer l'abondance d'une particule: elle nous renseigne sur la quantité d'états accessible mais reste à déterminer combien de particules résident sur un état donné. Le \textit{niveau d'occupation} dépend du type de particule: si celle-ci est un fermion alors elle est soumise au principe d'exclusion de Pauli qui stipule qu'un état quantique donné ne peut être occupé, au plus, que par une particule. Si celle-ci est un boson, cette restriction ne s'applique pas. Plus précisément, les niveaux d'occupation sont donnés par les statistiques de Fermi-Dirac et Bose-Einstein:
\begin{equation}
n(E)=\frac{g(E)}{\exp(\beta(E-\mu))\pm 1}.
\label{e:BEFD}
\end{equation}
Le signe positif (resp. négatif) au dénominateur désigne la statistique de Fermi-Dirac (resp. Bose-Einstein).La quantité $\beta=1/k_B T$ dépend est une représentation de la température, $\mu$ est le potentiel chimique de l'espèce étudiée et $g(E)$ est la dégénérescence d'un état d'énergie $E$. Cette dernière quantité dépend également de la particule considérée. On note que dans le cas d'une statistique de Fermi-Dirac, s'appliquant aux fermions, $n(E)\le g(E)$: cela découle du principe d'exclusion de Pauli: l'occupation est au mieux égale à la dégénérescence du niveau d'énergie. A l'inverse les bosons, soumis à la statistique de Bose-Einstein, peuvent avoir des niveaux d'occupation arbitrairement grands.

A ce stade, l'abondance d'une particule peut être déterminée et le nombre total de particules d'une espèce donnée dans une cuve de volume $V=L^3$ à température $T$ est
\begin{equation}
N=\int_{E_{\mathrm{min}}}^\infty n(E)N(E) dE.
\label{e:abondance}
\end{equation}
Notons que cette intégrale porte sur toutes les énergies, depuis la plus faible jusqu'aux infinis. Cette valeur plancher de l'énergie dépend de la particule considérée. Par exemple pour une particule de masse nulle, on aura $E_\mathrm{min}=0$ tandis que pour une particule massive on aura $E_\mathrm{min}=mc^2$, correspondant à un état de repos (et donc d'impulsion) absolu.

\section{Les photons}
Le cas des photons permet d'illustrer le calcul de la section précédente tout en étant d'une grande pertinence cosmologique: ils appartiennent aux particules dites relativistes et c'est elles qui dominent le budget numérique actuel de l'Univers. 

Le calcul de l'abondance des photons nécessite de préciser d'abord les quelques nombres nécessaires à sa bonne conduite. Dans un premier temps, les photons sont des particules de masse nulle, donc leur énergie est faite d'impulsion pure:
\begin{equation}
E_\gamma=pc=\hbar c k.
\end{equation} 
Par conséquent la densité volumique d'états d'énergie E accessible aux photons est donnée par:
\begin{equation}
\frac{N(E)dE}{V}=\frac{1}{2\pi^2}\frac{E^2dE}{(\hbar c)^3}.
\end{equation}
De plus le photon est sa propre antiparticule et participe par exemple aux équations de désintégrations:
\begin{equation}
A + \bar{A} \leftrightarrow \gamma+\gamma.
\end{equation}
Or $\mu_A=-\mu_{\bar{A}}$ donc $\mu_\gamma=0$. Enfin le photon autorise deux hélicités par état d'énergie et possède un spin entier et obéit donc à la statistique de Bose-Einstein. L'état d'occupation d'un niveau d'énergie E est donc donné par :
\begin{equation}
n(E)=\frac{2}{e^{\frac{E}{k_B T}}-1}.
\end{equation}
D'où son abondance à l'équilibre:
\begin{equation}
n_\gamma=\frac{1}{(\hbar c)^3\pi^2}\int_0^\infty\frac{E^2dE}{e^{\frac{E}{k_B T}}-1}.
\end{equation}
On reconnait dans l'intégrale la distribution de Planck, qui par définition décrit la distribution spectrale d'énergie d'un gaz de photons à l'équilibre, comme présent par exemple dans un corps noir. Cette intégrale peut être conduite analytiquement conduisant à :
\begin{equation}
n_\gamma \approx 0.244 \left(\frac{k_BT}{\hbar c}\right)^3 \mathrm{m}^{-3}.
\label{e:densphot}
\end{equation}

Aujourd'hui la température du gaz de photons du cosmos est de l'ordre de 2.73 K, correspondant à une densité de photons actuelle de :
\begin{equation}
n_\gamma\approx 410 \mathrm{cm}^{-3}.
\end{equation}
Pour mémoire, la densité d'atomes d'hydrogène actuelle (espèce qui domine la population de baryons) est de l'ordre de l'atome par $m^3$, on est donc dans un rapport de $10^{8-9}$,  extrêmement en faveur des photons. Une évaluation plus précise du rapport photon/baryon $\eta$ est donnée par:
\begin{equation}
\eta \approx 5\times 10^{-10}\frac{\Omega_b h^2}{0.02}
\end{equation}
Cette surabondance de lumière résulte du processus de désintégration des particules massives que nous étudieront par la suite. 

\section{Histoire de la Température}

Si l'on examine à nouveau l'expression de la densité de photon (Eq. \ref{e:densphot}), on constate que celle ci varie en $n_\gamma \sim T^3$. Ainsi si l'on considère une cuve de volume $V$ elle contient à un redshift $z$ donné le nombre de photons suivant:
\begin{equation}
N_\gamma(z) \sim V(z) T(z)^3.
\end{equation}
Or compte tenu de leur gigantesque domination numérique, ce nombre de photons doit être \textit{constant}: aucun processus (absorption/émission, à priori par des baryons extrêmement peu nombreux par rapport aux photons) ne peut changer $N_\gamma$ de façon significative. Or une cuve de taille donnée verra ses limites évoluer sous l'effet de la dynamique de l'Univers. En particulier $V=V_0 a^3$ d'où la loi d'évolution de la température des photons:
\begin{equation}
T_\gamma=\frac{T_0}{a}=T_0 (1+z),
\end{equation}
avec $T_0=2.73K$. De plus compte tenu de la domination quasi totale des espèces relativistes sur le bilan numérique des particules du cosmos (comme illustrée par la valeur de $\eta$), on peu presque considérer que cette température est celle du cosmos. Aujourd'hui l'Univers est froid, mais par le passé celui-ci était plus chaud, en plus d'être plus dense comme expliqué dans les chapitres précédents. Notons pour finir que les photons du cosmos ne sont plus à l'équilibre thermodynamique à proprement parler: aujourd'hui ces photons n'interagissent plus avec les baryons, interactions qui auraient permis de maintenir le bain de photons à l'équilibre. Toutefois dans le passé plus chaud et plus dense, ces interactions existaient et un régime de fort couplage permettait de garantir un couplage matière rayonnement suffisant pour que la situation "thermodynamique" de l'Univers s'apparente à celle d'un corps noir. Cette situation a cessé (380 000 ans après le Big Bang comme nous le verrons) mais la domination des espèces relativistes est telle qu'aucun processus n'est en mesure de changer significativement la fonction de distribution des photons: en l'absence de processus permettant cette modification, le gaz de photons a pu conserver la mémoire d'une période antérieure d'équilibre thermodynamique.

Par la suite nous considèrerons des époques durant lesquelles le bilan énergétique de l'Univers est dominé par les espèces relativistes durant lesquelles le facteur d'expansion varie en :
\begin{equation}
a\sim \sqrt t.
\end{equation}
Il en découle par la suite les lois d'échelles suivantes:
\begin{equation}
T\approx\frac{10^{10} K}{\sqrt{t\mathrm{(sec)}}} \approx \frac{1}{k_B}\frac{1 \mathrm{MeV}}{\sqrt{t\mathrm{(sec)}}}.
\end{equation}
Ces lois permettent déjà de se faire une idée des hautes températures en place durant les phases primordiales de l'Univers et donc permettent d'anticiper que des processus très énergétiques sont en mesure d'être effectifs. Par exemple les énergies typiques du LHC sont de l'ordre du TeV $=1e6$ MeV: elles correspondent aux énergies typiques dans un Univers de $10^{-12}$ secondes. De plus ces lois d'échelles permet d'anticiper que certaines époques joueront un rôle pour certaines particules quand l'énergie typique du cosmos est de l'ordre de l'énergie de masse des celles-là: par exemple l'électron possède une énergie de masse proche du MeV ( 511 keV exactement) et donc il est probable que son abondance soit significativement modifiée lorsque l'Univers aura un âge correspondant à cette masse (à savoir de l'ordre de la seconde).


\section{Evolution des abondances}
Pour une "particule" quelconque, l'expression précise de son abondance va dépendre de son caractère relativiste ou non. Il existe des particules pour lesquelles ce caractère reste inchangé au cours du temps, comme les photons par exemples, mais en général, une particule aura tendance à être considérée comme relativiste aux premiers instants de l'Univers puis évoluera plus tard vers le régime non-relativiste, avec comme conséquence une variation, parfois radicale, de son abondance au cours du temps.

L'énergie d'une particule libre est donnée par :
\begin{equation}
E=\sqrt{p^2c^2+m^2c^4}.
\end{equation}
 Une particule est dite \textit{ultra-relativiste} si son énergie de masse est considérée comme négligeable devant son énergie cinétique, $pc\gg mc^2$, auquel cas $E\sim pc$. C'est notamment le cas du photon, étudié en détail dans la section précédente. A l'inverse, une particule est dite non relativiste si son énergie de masse constitue l'essentiel de son énergie totale $pc \ll mc^2$. Cela correspond au fluide "matière" développé dans le chapitre précédent et dans ce cas $E\sim mc^2 (1+ p^2/2m) \sim mc^2$. L'utilisation de l'une ou l'autre de ces expressions pour l'énergie dans l'équation \ref{e:abondance} va conduire à des expressions différentes des abondances.
 
 \paragraph{Cas ultra-relativiste}
Ce cas correspond à celui étudié précédemment pour les photons : en effet, si l'on parle du principe que la masse d'une particule est négligeable, elle en devient quasi similaire à un photon et son abondance n'en diffère que par le facteur de dégénérescence et par la statistique à utiliser (BE ou FD). En conséquence, l'abondance d'une particule dans ce régime ultra-relativiste sera proche de celle des photons. Un calcul précis donne l'abondance suivante pour un \textit{boson}:
\begin{equation}
n_B=n_\gamma\frac{g_B}{2},
\end{equation}
tandis que si la particule étudiée est un fermion:
\begin{equation}
n_F=n_\gamma\frac{3g_F}{8}.
\end{equation}
A un facteur proche de l'unité près, l'abondance d'une particule relativiste est essentiellement celle des photons $n\sim n_\gamma$.

\paragraph{Cas non relativiste}
Dans ce régime, l'énergie d'une particule est la somme de l'énergie cinétique classique et de son énergie de masse, $E\sim p^2/2m +mc^2$ et l'occupation statistique des énergies (cf. eq. \ref{e:BEFD}) devient la statistique de  Maxwell-Boltzmann. Le calcul de son abondance donne une abondance qui dépend directement de la température:
\begin{equation}
n=ge^{\frac{\mu}{k_BT}}\left(\frac{m k_B T}{2\pi\hbar^2}\right)^{3/2}e^{-\frac{mc^2}{k_B T}}\sim e^{-\frac{mc^2}{k_B T}}
\label{e:nonrel}
\end{equation}
Compte tenu de l'évolution de la température, qui décroit avec le temps, l'abondance décroît de façon exponentielle. La réaction typique permettant cette décroissance est une réaction de \textit{désintégration}:
\begin{equation}
A+\bar A \leftrightarrow \gamma+\gamma,
\end{equation}
avec un déplacement de l'équilibre vers la droite de cette équation, i.e. vers le réservoir de photons.

\paragraph{Transition}
Se pose alors la question de la détermination du régime dans lequel se trouve une particule. Il s'avère que la température d'un gaz de particule est liée à l'énergie cinétique et $E_c\sim k_B T$. Par conséquent à haute température, $k_B T\gg mc^2$, une particule tend à être ultra-relativiste tandis qu' à basse température $k_B T\ll mc^2$, celle-ci tend à être non relativiste. On sait également que la température de l'Univers décroît au cours du temps, donc pour une particule massive donnée, il se trouvera toujours une époque reculée où cette particule est relativiste, suivie par une époque où elle basculera dans le régime non-relativiste. La transition entre les deux régimes opère lorsque l'énergie cinétique typique est de l'ordre de l'énergie de masse:
\begin{equation}
k_B T(z^*)=mc^2.
\end{equation}
Sachant que la température du rayonnement varie en $T_\gamma\sim (1+z)$ et qu'à l'équilibre un fort couplage existe, la transition opère à un redshift $z^*$ donné par :
\begin{equation}
1+z^*=\frac{mc^2}{k_B T_0}.
\end{equation}
 On constate ainsi qu'une particule passera dans le régime non relativiste d'autant plus rapidement qu'elle sera massive. A l'inverse, une particule de masse nulle ne pourra jamais, comme attendu, basculer dans le régime non relativiste.
 

 
 \section{Abondances résiduelles}
  En résumé, l'abondance à l'équilibre d'une particule passe par 2 étapes distinctes:
 \begin{itemize}
 \item à grand $z>z^*$, nous avons $k_B T \gg mc^2$ et $n\sim n_\gamma$
 \item à bas  $z<z^*$, nous avons $k_B T \ll mc^2$ et l'abondance décroît de façon exponentielle. Elle se désintègre et son abondance devient très inférieure à l'abondance des photons, $n\ll n_\gamma$.
 \end{itemize}
Or nous avons vu précédemment que les réactions qui permettent le maintien de cet équilibre vont "geler" et découpler une espèce de la "soupe" de particules en interaction: après ce gel, l'abondance d'une espèce va rester celle de l'équilibre au moment du découplage. Ce gel peut opérer avant ou après $z^*$. Si une particule gèle pour $z>z^*$, elle se trouvait dans son régime relativiste, en grande abondance. Un tel type de particule va rester très abondante jusqu'à nos jours et c'est par exemple le cas des neutrinos. Si à l'inverse elle gèle pour $z<z^*$, alors celle-ci avait déjà entamé sa désintégration durant laquelle son abondance décroît de façon exponentielle. Par conséquent la particule est en très faible abondance et aujourd'hui son abondance doit être très faible devant celle des photons (et donc des neutrinos). C'est le scénario essentiellement de toutes les particules massives, que l'on dit aussi \textit{particules reliques} car elles auront survécu à la désintégration.
 