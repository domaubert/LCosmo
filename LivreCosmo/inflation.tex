\chapter{L'inflation et les premiers instants}
\newthought{L'inflation} désigne une phase de l'histoire de l'Univers durant laquelle l'espace aurait rapidement 'enflé', avec des taux d'expansion exponentiels. Cet épisode aurait pris place environ $10^{-34}$ secondes après le Big-Bang et a été suggéré à partir des années 1980 pour expliquer toute une série de défis qui se posent quant à l'état de l'Univers tel que nous l'observons. L'époque inflationnaire reste pour l'instant une hypothèse non vérifiée expérimentalement mais pour laquelle existe tout un faisceau de présomption.

\section{Le problème de l'Horizon de causalité}
\newthought{Le principe cosmologique} repose sur une hypothèse d'homogénéité de l'Univers et cette homogénéité n'est pas remise en question aujourd'hui par les observations. L'une des manifestations les plus spectaculaire de cette homogénéité est la température du fond diffus cosmologique~: comme expliqué dans le chapitre dédié, le fond diffus cosmologique présente une température typique $T\sim 2.7 $K à un très haut niveau de précision (à $10^{-5}$ près) et ceci quelle que soit la direction vers laquelle on regarde \sidenote{on rappelle que c'est cette isotropie qui intrigua Penzias \& Wilson lors de leur découverte du signal}. 

Des anisotropies existent mais elles sont noyées dans l'amplitude du signal du monopole~ : on peut citer l'empreinte des oscillations baryoniques accoustiques, déclenchées par la compétition entre gravité et pression de rayonnement, qui produisent ces grands pics dans le spectre de puissance angulaire du fond diffus. En particulier, les plus grandes échelles angulaires associées à ces ondes accoustiques sont de l'ordre du degré sur le ciel \sidenote{correspondant à une fréquence angulaire $\ell\sim 1000$}. Cette plus grande échelle angulaire correspond à \textit{l'horizon sonore} aux époques de l'émission du fond diffus~: cet horizon est la plus grande distance qui peut être parcourue à la vitesse du son dans les conditions qui y régnaient. Cette vitesse du son est régie par la pression du rayonnement et est de l'ordre de :
\begin{equation}
c_s\sim \frac{c}{\sqrt 3}.
\end{equation}
tandis que l'horizon peut-être approximé par:
\begin{equation}
L_H\sim\frac{c_s}{H}
\end{equation}
On constate donc aisément que l'horizon sonore est proche de l'horizon causal, déterminé lui par la vitesse de la lumière.

Nous avons donc une surface de dernière diffusion qui est isotrope à un très haut niveau de précision \textit{sur toute la sphère céleste} tandis que les échelles de longueurs en contact causal (donc de taille inférieure à l'horizon) sont particulièrement ramassées. Par conséquent, on ne peut trouver de processus physique qui soit en mesure de propager une information sur tout le ciel 380 000 ans après le Big-Bang : cette homogénéité et isotropie ne peut trouver son origine dans un mécanisme physique qui aurait établi ces propriétés sur ces très grandes échelles sans lien causal.

\newthought{Deux possibilités} s'offrent à nouveau : cette isotropie est une condition initiale, particulière mais établie sans raison aucune ou bien cette isotropie est bien le fruit de la propagation d'un signal physique mais sur des échelles plus faibles que celles sur lesquelles l'isotropie est aujourd'hui observée. L'inflation intervient dans ce second scénario : une théorie de l'inflation stipule que l'on ne peut extrapoler l'histoire d'expansion de l'Univers vers le Big-Bang à partir de son contenu actuel et donc de sa dynamique actuelle. Il faut invoquer un épisode où le paramètre d'expansion $a(t)$ connaît une variation soudaine, faisant traverser l'horizon à des échelles en initialement en lien causal. L'idée est simple~: les plus grandes échelles observées sur le ciel étaient sous l'horizon avant l'épisode inflationnaire et sont passées hors-horizon après ce dernier. 

Compte tenu des échelles en jeu\sidenote{on rappelle que l'isotropie est observée sur toute la surface de dernière diffusion, dont le rayon est de l'ordre de plusieurs Gpc}, cette inflation doit impliquer des croissances gigantesques. On verra par la suite que les distances doivent s'accroître typiquement d'un facteur $10^{50}-10^{60}$.

\section{Le problème de la platitude}
\newthought{La géométrie de l'Univers est plane} en moyenne et sur des distances cosmologiques. Dans une telle géométrie, la lumière se propage en ligne droite et la somme des angles d'un triangle fait 180 degrés : comme expliqué dans le chapitre dédié, la taille angulaire des osciallations baryoniques accoustiques mesurée dans le fond diffus à $z\sim1100$ ou dans les grands relevés de galaxies à $z\sim 0$ est hautement compatible avec ce type de géométrie. Un Univers sphérique a tendance à surestimer ces tailles angulaires, un Univers hyperbolique à les sous-estimer et de fait la réalité terrain semble indiquer que le régime à l'oeuvre est exactement entre ces deux régimes.

Tout comme le problème de la causalité, le fait d'avoir une géométrie plane peut soit être le fruit d'un mécanisme qui aurait aplati une géométrie arbitraire ou bien la conséquence d'un choix de conditions initiales particulier. Et à nouveau cette seconde option n'est pas entièrement satisfaisante bien que tout à fait possible. En proposant une augmentation exponentielle des distances, l'inflation fournit naturellement un mécanisme pour gomme toute sorte de courbure~: l'Univers était peut-être doté d'une courbure non-nulle à une époque reculée mais l'Inflation aurait fait tendre toute courbure initiale non nulle vers zéro.

\section{L'origine des fluctuations cosmiques}
\newthought{Les grandes structures de l'Univers} trouvent leur origine dans l'existence de fluctuations dans la distribution spatiale de la densité d'énergie (ou de matière). Ces fluctuations sont observées à de très faibles niveaux dans le fond diffus cosmologique et ce sont ces 'graines' qui servent de point d'ancrage au processus d'instabilité gravitationnelle. Sans ces fluctuations, pas de structures dans l'Univers actuel. 

Comme indiqué précédemment, on attend d'une période inflationnaire qu'elle conduise à un accroissement des échelles de longueurs d'un facteur $\sim 10^{55}$. Si l'on prend une structure de taille 10 Mpc aujourd'hui, un tel facteur conduit à  une taille initiale de l'ordre de $10-{20}$ m, c'est à dire des échelles très largement soumises à des processus quantiques. Par conséquent, on peut imaginer que les structures observées aujourd'hui sont le fruit du passage de fluctuations quantiques à l'échelle macroscopique, par le biais de l'Inflation.