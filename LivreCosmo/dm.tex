\chapter{La Matière Noire}

Dans le modèle de la cosmologie standard, la matière est composée à environ $80\%$ d'une matière invisible sans interaction avec le rayonnement électromagnétique~: on la désigne sous le nom de \textit{matière noire}. Cette matière est nécessaire pour expliquer notamment la distribution de matière aux plus grandes échelles, telle qu'elle est sondée par exemple par le fond diffus cosmologiques ou les grand relevés de galaxies. Elle est de fait un ingrédient essentiel du processus de formation des grandes structures de l'Univers, étudié plus en détail dans un chapitre dédié. Mais avant d'étudier cette formation, nous allons dédier un court chapitre aux indices en faveur de l'existence de cette matière et les quelques propriétés dont on pense qu'elle est pourvue. Cette matière n'est toutefois pas sans poser problème, en particulier aux échelles galactiques~: on décrira quelques-uns de ces problèmes ainsi que les pistes possible d'une réduction des tensions que pose cette matière noire.

\section{Matière noire et dynamique interne des structures}
L'une des indications les plus fameuses de l'existence de cette matière noire est la différence quasi-systématique entre la dynamique interne observée des grandes structures (galaxies, amas de galaxie) et celle prédite par son contenu lumineux.

\subsection{Courbe de rotation des galaxies}
L'exemple le plus connu est celui de la courbe de rotation plate des galaxies. La courbe de rotation désigne  la façon dont la vitesse de rotation de la matière dans un système auto-gravitant varie en fonction de la distance au centre de ce système. Par exemple, considérons une masse $M$ ponctuelle : un corps en orbite circulaire de rayon $r$ aura une vitesse de rotation donnée simplement par :
\begin{equation}
V_r(r)=\sqrt{\frac{GM}{r}}.
\end{equation} 
Ce comportement en $1/\sqrt{r}$ est par exemple celui observé dans le système solaire, où les corps les plus éloignés du Soleil sont aussi ceux qui orbitent le plus lentement. Pour un système étendu avec un profil de masse la relation reste inchangée:
\begin{equation}
V_r(r)=\sqrt{\frac{GM(<r)}{r}}.
\end{equation}
C'est la masse comprise à l'intérieur de l'orbite qui rentre en jeu : celle-ci peut augmenter avec le rayon\sidenote{par exemple un système de densité homogène voit $M(<r)\sim r^3$  et donc $V_r\sim r$ }, mais si il existe une distance au delà de laquelle cette masse ne varie plus \sidenote{comme attendu pour un système de taille finie}, on retrouve la décroissance standard en $1/\sqrt{r}$ de la vitesse de rotation.

On peut faire le m