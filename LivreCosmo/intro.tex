\chapter*{Introduction}
\newthought{On ne mesure pas assez la chance} de pouvoir faire de la cosmologie. Cette possibilité nous est offerte aujourd'hui car nous disposons de moyens instrumentaux capables de sonder l'Univers le plus distant et le plus ancien, car nous disposons de théories et modèles suffisamment matures et également parce que la nature aura bien voulu se laisser observer, se présentant dans un état d'évolution tel que le cosmos intéressant est aujourd'hui accessible. 
L'objectif de ce livre est donc de célébrer cette opportunité qui nous est donnée.

La cosmologie se fixe pour objectif démesuré, délirant voire mégalomaniaque d'étudier l'Univers dans sa globalité comme on étudierait une ville, un espèce dans son habitat ou un système de pensée philosophique. On se dit qu'on peut le faire, osant dire 'chiche' devant l'énormité de la tâche. L'Univers  est actuellement en train d'évoluer, on ne peut le reproduire en laboratoire, nous faisons partie de l'expérience, il ne nous est pas entièrement inaccessible et pourtant l'ambition de la cosmologie est de passer outre ces obstacles et de faire au mieux pour pouvoir comprendre, prédire et décrire ce qui contient tout ce qui est, tout ce qui est passé et tout ce qui adviendra.

Ce que cet ouvrage va tenter de montrer, alors que tout semble montrer que cette tâche est impossible à priori, c'est que la cosmologie est viable~: elle fonctionne, en dépit du bon sens. L'édifice des théories physiques développé au cours des siècles parvient autant que faire ce peut à dégager un modèle cohérent, relativement simple de notre Univers. Nous avons affaire à un objet naturel, le cosmos, que la science permet d'aborder sous de multiples aspects et chacun de ses aspects produit une vision globale de notre Univers et de son évolution. La cosmologie étudie un objet avec tous les outils de la physique dont nous disposons, plutôt que de s'attacher à une théorie physique et l'appliquer aux multiples contextes où elle peut être pertinente : la cosmologie est une discipline de l'astrophysique, c'est à dire une science naturelle. C'est pour cette raison que ce livre est davantage un panorama de divers aspects de la cosmologie, sur la richesse physique nécessaire à sa description, plutôt qu'un ouvrage totalement cohérent et autonome de l'état actuel de notre compréhension de l'Univers. Il se passe beaucoup de choses dans notre Univers et l'emphase est mise sur la diversité des modèles physiques qu'il faut invoquer pour tenter de comprendre toutes ses facettes.

Une autre des motivations qui pousse cet ouvrage à être écrit ainsi est la conviction que l'astrophysique (et la cosmologie en particulier) est le lieu idéal pour promouvoir la physique, sa richesse, sa capacité à décrire les régimes naturels les plus inattendus. En particulier, la physique est traversée de démarches constantes et similaires appliquées dans les contextes les plus divers~: la cosmologie permet de mettre cette versatilité en avant. De façon récurrente, il est des régimes d'approximations ou d'extrapolations qui permettent à des théories physiques trop complexes de prime abord de décrire simplement et précisément des processus à l'oeuvre dans le cosmos. On réalise ceci progressivement, avec l'expérience, avec les confrontations multiples à l'étude de cas physiques divers et variés : l'objectif de ce livre est aussi d'amorcer cette prise d'expérience pour le lecteur.

Enfin, l'astrophysique et la cosmologie ont besoin des mathématiques~: c'est une évidence qui ne peut et ne doit pas être niée. Pour cette raison, ce livre a un biais théorique et affiche un grand nombre d'équations. Ceci étant posé, je considère que les mathématiques ont plus besoin de l'astrophysique que l'inverse et de fait je pense également que la physique en général permet d'appréhender, de façon intuitive, comment est structurée une équation, comment des quantités pertinentes peuvent être anticipées. La physique permet une approche empirique des mathématiques, qui dans bien des cas permet de faciliter leur compréhension et démontrer en quoi elles sont utiles et dans quels contextes.

\newthought{La cosmologie fonctionne} mais comme montré dans cet ouvrage, cela a un coût. Ce coût est d'une part la perte de certaines intuitions façonnées par notre expérience quotidienne~: la perte d'une définition absolue de l'espace en est l'exemple le plus frappant. Par l'expérience, c'est toute une habitude, une façon de penser qui doit être rapidement prise par le lecteur~: cet ouvrage cherche aussi dédramatiser cet aspect et par extension à tordre le cou à certaines idées reçues ou mal expliquées. D'autre part, le modèle standard de la cosmologie a un coût important en ce qu'il nous impose d'admettre notre grande ignorance. Par exemple 95\% du bilan énergétique actuel de l'Univers nous est simplement inconnu\sidenote{par exemple environ un quart de cette énergie est sous forme de matière noire et environ deux tiers est sous la forme d'énergie noire}~: l'influence de ce 'secteur sombre' est parfaitement décrite par nos théories, rencontre grand nombres de succès, mais sans que sa nature nous soit accessible. C'est la grande ombre qui plane sur cette discipline, telle l'épée de Damoclès~: toute difficulté rencontrée par ce modèle nous force à lever la tête vers cette incertitude permanente, vers la pensée lancinante que cet échaffaudage n'est pas forcément des plus stables. Et pourtant bien des choses sont trop belles pour ne pas être vraies. C'est pour cette raison que ce livre cherchera en permanence à mettre en regard succès et faiblesses de la cosmologie moderne.

Les thèmes qui y sont abordés sont relativement standards, tout comme son organisation. Les premiers chapitres sont dédiés à l'Univers parfait, homogène et à sa dynamique afin d'établir sa caractéristique la plus spectaculaire à savoir \textit{l'expansion}. Cette expansion, prédite par la théorie de la relativité générale donne à notre Univers un début, une histoire et par extension un avenir que nous sommes capables de lui prédire. Par la suite, l'ouvrage se focalise sur l'Univers à ses débuts, lorsque les température sont élevées et les densités importantes~: c'est le royaume de la cosmologie primordiale, de la physique des énergies inaccessibles aux accélérateurs actuels. La conclusion de cette phase est la production\sidenote{380 000 ans après le Big-Bang} de l'objet le plus important de la cosmologie, le fond diffus cosmologique, et peut-être l'objet naturel le plus important de toute la physique~: étudié sous toutes ces coutures, compris à des degrés rarement égalé dans d'autres disciplines, le fond diffus est presque une anomalie dans le paysage de l'astrophysique. Sa compréhension est maîtrisée et son importance mainte fois démontrée et c'est probablement la source la plus importante de notre savoir actuel sur l'Univers. Les chapitres suivants sont axés sur l'Univers hétérogène et la formation des structures, la thématique de la matière noire, ou la réionisation du milieu intergalactique~: ces sujets sont le pont entre la cosmologie simple et pure et l'astrophysique complexe. C'est le fondement de la théorie de la formation des galaxies, sujet qui ne sera pas abordé en détail dans ce livre, car il mérite plusieurs ouvrages en soi \sidenote{et de multiples références existent d'ores et déjà}. Toutefois, un chapitre est dédié aux simulations cosmologiques, programmes informatiques qui visent à modéliser cette cosmologie hétérogène dans toute sa complexité~: ce chapitre est à nouveau l'occasion d'exposer la variété des physiques pertinentes à la cosmologie et c'est également un sujet rarement abordé dans des livres équivalent. Comme il s'agit de mon sujet de recherche principal, ce chapitre est l'occasion de présenter une expertise qui m'est chère. On m'excusera par avance de ne pas dédier de chapitre aux moyens d'observations équivalents~: ces moyens et les techniques associées sont généralement décrits le long des différents chapitres lorsque cela est nécessaire et par ailleurs, il existe nombre de livres de grande qualité dédiés à l'observation astronomique\sidenote{la bibliographie donnée en fin d'ouvrage mentionne de tels ouvrages} qui seront bien plus utiles au lecteur que ce que j'aurai pu rapidement proposer ici. L'ouvrage se termine sur des sujets plus spéculatifs ou plus abstraits, comme l'inflation ou le rôle de l'entropie en cosmologie~: c'est à nouveau l'occasion de mettre l'emphase sur la variété d'outils physiques qu'il faut mobiliser pour avoir une compréhension de notre Univers.

\newthought{Le public visé} par ce livre est celui d'étudiantes et d'étudiants de filière scientifiques, de Licence, de Master et d'écoles d'ingénieurs. Le pré-requis sont ceux d'une maîtrise de l'analyse et de l'algèbre de terminale scientifique, même si l'ouvrage essaie autant que faire ce peut de présenter différents niveaux de lecture et fournir des explications ou des conclusions qui ne nécessite pas forcément de suivre les démonstrations mathématiques. De fait, je tiens à préciser que ce livre n'est pas complet, parfois imprécis et prend sans honte un certain nombre de raccourcis~: il existe de multiples références qui sont elles complètes dans leurs démonstrations, exhaustives dans leurs analyses parfois au prix d'une complexité élevée. Ici, on vise davantage à exposer les démarches, les idées en sacrifiant ici et là à des simplifications. De fait, cette démarche reflète ma propre approche de la discipline~: je ne suis pas moi-même un spécialiste de cosmologie à proprement parler, mais mon activité de recherche centrée sur la production de modèles où le contexte cosmologique est extrêmement important m'a conduit à développer une connaissance large, empirique de cette discipline. C'est ce rassemblement d'idées, d'approches, de compréhensions parfois superficielles mais dont l'utilité m'a été démontrée par ma pratique de la science que je souhaite partager dans cet ouvrage. De fait, j'ose espérer que cet ouvrage intéressera le non-physicien, voire le non-scientifique mais toute personne intéressée par la mise en hypothèse, le raisonnement, la conceptualisation.

\newthought{Cet ouvrage doit beaucoup} à mes collègues de l'Observatoire Astronomique de Strasbourg et d'ailleurs, à mes étudiants et aux mentors qui ont tous façonné d'une manière ou d'une autre ma trajectoire de scientifique et d'enseignant à l'Université. Je tiens ainsi à remercier Pierre Ocvirk, Jonathan Chardin, Nicolas Gillet, Nicolas Deparis, Joe Lewis, Sébastien Derrière, Arnaud Siebert, Caroline Bot, Thomas Keller, Thomas Boch, Sandrine Langenbacher, Romaric David, Medhi Amini, Nicolas Martin, Ariane Lançon, Christian Boily, Rodrigo Ibata, Benoit Famaey, Jean-Marie Hameury, Olivier Bienaymé, Hervé Wozniak et Pierre-Alain Duc. La contribution implicite de Christophe Pichon et de Romain Teyssier à cet ouvrage est incommensurable. Ce livre doit également beaucoup à certaines rencontres dont celles avec Benoit Sémelin, Jérémy Blaizot, Mathieu Langer, Marian Douspis, Yann Rasera, Mathias Gonzalez, Edouard Audit, Françoise Combes, Francis Bernardeau, Nabila Aghanim, Guilaine Lagache, Jean-Loup Puget, François Bouchet, Stéphane Colombi, et Stéphane Basa. Pour finir je tiens à remercier Anne-Catherine, Anouk \& Raoul, mes phares immuables au sein de ce cosmos infini.